% --------------------------------------------------------------
% This is all preamble stuff that you don't have to worry about.
% Head down to where it says "Start here"
% --------------------------------------------------------------
 
\documentclass[12pt]{article}

% Code snippets
\usepackage{xcolor}
\usepackage{listings}
\lstset{basicstyle=\ttfamily,
  showstringspaces=false,
  commentstyle=\color{red},
  keywordstyle=\color{blue},
  breaklines=true,
  postbreak=\mbox{\textcolor{red}{$\hookrightarrow$}\space},
}
 
\begin{document}
 
% --------------------------------------------------------------
%                         Start here
% --------------------------------------------------------------
 
\title{Secure Workflow Demo}

\maketitle

This demo has been realized using Istio-1.2.0, Kubernetes-1.15.0 and minikube-1.2.0.
The demo has also been tested with Istio-1.3.3, Kubernetes-1.16.0 and minikube-1.2.0.

\section{Creating the services}

The first step is the creation of the workflow services.
We have created two REST APIs, \textit{workflow-adder} and \textit{workflow-multiplier}, which can perform respectively additions and multiplications on numbers passed with a PUT request.
The numbers and the result of the operation are then stored in the running service, and can be queried with a GET request.
See the \textbf{services} folder for more details on the implementation.

\section{Containerizing the services (Docker)}

The second step is to turn those services into their containerized versions.
The \textit{Dockerfile} and the \textit{requirements.txt} files are added to do this.
We add an owner service to simulate the owner of the workflow.



\section{Installing Kubernetes}

Prerequisites: Install KVM or VirtualBox:
\begin{lstlisting}[language=bash]
sudo add-apt-repository multiverse && sudo apt update
sudo apt install virtualbox
\end{lstlisting}


\begin{lstlisting}[language=bash]
# Make sure vmx or svm is enabled
egrep --color 'vmx|svm' /proc/cpuinfo

# Install kubectl
curl -LO https://storage.googleapis.com/kubernetes-release/release/$(curl -s https://storage.googleapis.com/kubernetes-release/release/stable.txt)/bin/linux/amd64/kubectl
chmod +x ./kubectl
sudo mv ./kubectl /usr/local/bin/kubectl
kubectl version

# Install minikube
curl -Lo minikube https://storage.googleapis.com/minikube/releases/latest/minikube-linux-amd64 && chmod +x minikube
sudo install minikube /usr/local/bin

# Start minikube. You might need a restart before
minikube start --memory=8192 --cpus=4
\end{lstlisting}



\section{Installing Istio}

\begin{lstlisting}[language=bash]
# Install Istio
curl -L https://git.io/getLatestIstio | ISTIO_VERSION=1.3.2 sh -
cd istio-1.3.2
export PATH=$PWD/bin:$PATH

# Install the Istio Custom Resource Definitions (CRDs)
\begin{lstlisting}[language=bash]
for i in install/kubernetes/helm/istio-init/files/crd*yaml; do kubectl apply -f $i; done

# Enforce strict mTLS authentication
kubectl apply -f install/kubernetes/istio-demo-auth.yaml

# Enable automatic injection of the Envoy proxies for services of the workflow
kubectl label namespace default istio-injection=enabled

# Check the installation
kubectl get pods --all-namespaces
kubectl get svc --all-namespaces
\end{lstlisting}



\section{Adding Open Policy Agent}

\begin{lstlisting}[language=bash]
# Install the OPA-Istio CRDs and our custom policy
kubectl apply -f custom_quick_start.yaml

# Enable automatic injection of the OPA sidecars for services of the workflow
kubectl label namespace default opa-istio-injection="enabled"
\end{lstlisting}



\section{Deploying the workflow}

\begin{lstlisting}[language=bash]
# Use the same \textbf{Docker} host as the minikube VM
eval $(minikube docker-env)

# This command can be used later to undo this:
eval $(minikube docker-env -u)

# Build the service images
docker build -t my_docker_adder:latest .
docker build -t my_docker_multiplier:latest .
docker build -t my_docker_owner:latest .

# Check if the images are in minikube's Docker registry
minikube ssh docker images

# Deploy the workflow services
kubectl apply -f workflow-deployment.yaml

# Check if the services were deployed:
kubectl get deployments
\end{lstlisting}


% --------------------------------------------------------------
%     You don't have to mess with anything below this line.
% --------------------------------------------------------------
 
\end{document}
